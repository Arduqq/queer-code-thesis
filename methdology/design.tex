\section{Design}
% Why is this a probe study?
The main study is conducted as an inspirational probe study.  As described by \citep{mattelmäki:2005}, my probes are of inspirational type and are meant to provoke experiences that may enable open reflection on the sensitive, social topics of computing in society. For this research, my probes are defined as such by its fundamental properties. A kit consisting of a zine, a website with a personal participants' place, and an editor give room for exploratory opportunities the encourage individual and ambivalent confrontation with the topic. For the probes are meant to showcase the creative possibilities for both coders and educators, the study sets an exploratory goal of rethinking the role of computing outside the scope of the industry. By moving the study into learning environments chosen autonomously by the participants, the probes become a part of the users' subjective world as the zine may be taken anywhere, and programming tasks are encouraged to be personal and undefined. The resulting program collages are artifacts of the self-documentation process of reading, note-taking, designing, and reflecting. The computer as a medium for creative self-documentation still remains overshadowed in probe studies by pen and paper, tablets, or mobile input technology. The internet itself as a medium for self-expression provides opportunities for users to log their experiences with probes.
% Why probes?
Choosing this form of a free, yet tangible concept is motivated by the active involvement of the users. Giving participants the building blocks to explore supplementary material like a zine alongside a digital canvas, may enforce more intimate and playful experiences compared to a sequential completion of artificial problems. The open nature of probes encourages strategizing a personal workflow and treats a programming course not as a lecture of topics, but an exploration. As an opportunity to move away from the linear style that sees its implementation in modern day programming courses, probe studies leave room for creative involvement that may even give participants a sense of pride in the creative work they produced. Elevating the probe from a solely analogue form to a digital workspace, may provide further insight on the creative potential of digital media in a playful context and the definition of probe studies. Where digital programming courses lack in tangibility of the code work, the probe can be seen as a gift and real representation of their efforts. \\ \\
% How to overcome the problems of probes? 
As noted by \citeauthor{mattelmäki:2005}, probe studies come with their own difficulties as their ambiguity may frustrate and confuse participants. Providing users with numerous ways of interacting with the probes, like note-taking in the zine, clicking through resources, using different building blocks for the collages, and finding ways to overcome its limitations, may give questioning users of catching momentum when trying out the different interactions. Furthermore, removing pressure in the form of strict deadlines may alleviate stress. Providing inspirational aid at all times to varying extent while having initial rapport with participants may also help users to overcome the hurdle to create freely.