
\section{Procedure}
The study is communicated both in person and through online communication over e-mail. In a personal meeting, each participant is handed their probe kit and is instructed in the details of the study. In a collective message, the details are again described and the website is shared. During the study, participants can message me personally in case any technical difficulties come up. In case creative problems arise, I provide further aid in form of pseudocode examples and further explanations on the topics. In case of time issues, the individual study deadline of two weeks are postponed. 
Participants are presented with eight topics of computer programming, and will create at least five art pieces through the graphical interface on the website. Over the course of a minimum of two weeks, they can read the information in the zine, unlock the digital resources, and create a code piece of whatever topic they wish. For additional inspiration, questions are provided as thought impulses. \\ \\
New content on the website is unlocked by the users as they progress the reading of the zine. By typing a new code on the platform, digital resources and open questions of interpretation are unlocked that are supposed to spark inspiration to the participants. Largely inspired by the resources provided in the Aesthetic Programming course, the references are realized in simple button clicks alongside a short preface. Instead of researching links from a long list of literature, the more visual and easily digestible content provides participants with an easier experience. Accordingly, they may invest their time to their liking and explore the things they find the most interesting instead of reading through long articles without personal connection. \\ \\
The additional resources are used as inspiration for their programming task and will then navigate to the respective editor to start implementing. User may submit their work over their hub, and will see the progress in the form of their finished work. Every submission comes with a short description of the participant’s vision, their artistic choices, and a reflection on the hypothetical power dynamics of the program. The submissions are then showcased to the participants over their hub. \\ \\ 
To provide context to the art the participants produce, a short assessment of their experience and opinion with programming is conducted in form of a survey. Here, the overall demographic, their societal perception of programming, and current knowledge on the topic is asked to aid the analysis of the probe study. \\ \\ 
A supporting interview will be conducted in a semi-structured form where participants will be asked about their experience with the learning platform. If they are comfortable, participants may talk about the code pieces they created to assess their thought processes, and whether their ideas were realized on the learning platform accordingly. Talking to participants directly will also provide valuable information on whether the course was able to communicate the topics appropriately. \\ \\
Interviews are conducted in a personal meeting in a familiar setting like my or their home. The interview is recorded on audio and is then transcribed to analyze the participants’ opinion on the information provided, and their learning efforts.
For the analysis, the graphical interface’s import functionality is used to view the code pieces created by the participants. The tool enables the deduction of what features were especially useful for conveying certain types of programming concepts. The provided self-reflection by users facilitates the interpretation of the results. Findings are then finalized through individual questions during the interview.