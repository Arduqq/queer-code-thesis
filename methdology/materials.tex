\section{Materials}
A probe study is valuable to the analysis of a learning platform, as it encourages the participants to partake in the numerous resources and explorable environments provided. Reading articles, unlocking custom content and creating playful designs is supposed to aid their willingness to learn in a non-exploitative context. In the sense of material feminism, getting involved with the medium physically is crucial to learning experiences. Transferring participants between analogue and digital media supports the idea that programming is not an exclusively digital craft. Print media gives room for experimentation, note-taking and drawing, whereas digital interfaces provide aesthetic feedback that encourages exploration.
\subsection{Mixed-Media Probe Study}
% What does the probe look like?
The probe is both analogue and digital in form of a written zine containing the learning articles, and a hub space on a website to track your progress. To give the platform recognizability and a homely feeling, the design of the probe is cohesive throughout and is given character through custom graphics, icons and style. The title Assemblyre is a direct reference to the concept of the Assembler, a programming language that directly transforms assembly language to computer language, and a lyre, which puts emphasis on the effort of highlighting the inherent creative potential of computing and its acceptance. The motivation to change between analogue and digital comes from the reference of codes throughout the zine. User codes remain unique and serve as a way to identify the work and their submitted task, while the codes of the courses are the same since there are no variations of the respective chapters.
\subsubsection{Programming Course Design}
% What is part of the course?
Every chapter consists of the excerpt from the guide on Aesthetic Programming by \cite{soon:2020}, the respective course ID code, open questions about the topic discussed, and a list of links to digital resources with short summaries. The overall courses encompasses the text material, thought-provoking questions, and the curated list of online resources. The extent of the text was chosen with its educational value in mind without overwhelming participants with new information for the chosen duration of the study. Driven by the multi-medial nature, I chose print media for the reading portion to prevent long screen times, and digital websites for further resources to bring online sources to people's attention that only work when viewed through a Browser. This notion underlines the medial nature of computing and the web as it enables expression that couldn't exist anywhere else. The material approach to programming is supposed to be reflected in the way participants interact and get involved with the media outside of the necessity to code conventionally using a computer with an IDE.
\paragraph{Text Collection}
% How were course chapters selected?
The excerpts were chosen according to the scope of this study. To avoid confusion, programming tasks with P5js, reflections on the original Aesthetic Programming course, and paragraphs relying on the involvement in it have been removed to serve the purpose of being publishable in a zine. With that in mind, I cited their work without obfuscating meaning and provided context where it may have been unclear. The goal of the zine would not be the active confrontation with the complex details, but the acknowledgment of their existence, to spark interest and leave the investigation up to the participants. [ref] visualizes the changes and adjustments I made to use the course for the scope of the study as described above. [#1]
\paragraph{Resource Collection}
Gathering resources to share with the participants required curating of commented references in the Aesthetic Programming course while exploring their references and sources. With the help of the Cyberfeminism Index \citep{cfi} and the video game repository of Itch.io \citep{itch}, the repository grew to offer a multi-faceted experience. Alongside short and long documentations, participants may look at art galleries and play video games supporting the topics of the course chapters. This easily digestible content is meant to narrow the gap between participants and the topic. Participants may enjoy the content, recognize the creative work of others in the field, and get accustomed to computing as something not inherently out of their reach.
\paragraph{Course Structure}
To fit the course to the duration of the study while including every topic relevant to the problem, topic selection remained close to the source material. Since not all details of every chapter are a part of the study, brief references through online resources reference the broad context to not dismiss it. The topics discussed in their chapters are explained in [ref]. [#2] \\ \\
% Warum 5/8
Each participant is meant to create five collages based on the chapters they found most interesting. While there are eight chapters to read about, only five of them are required to submit entries for. It's apparent that not every topic will appeal to everyone, and in the context of a probe, participants are meant to explore the topics and give their insight on the ones most intriguing to them. It is to say that chapters are not identical in their length and may give participants not comfortable with a topic pursue a different one. The study strives to bring people closer to the field of computing by building bridges closer to the technical field without choice paralysis or creative block. That is why a course structure with a clear line of sight while giving control to the user seemed helpful for the cause.
\subsubsection{Analogue Zine}
The zine provided during the study is the very first probe participants come in contact with. Designed in Adobe Illustrator and printed as an DIN A6 brochure, it is supposed to be portable and interesting to look at. Its design resembles the editor and takes inspiration from zines in the cyberfeminist sphere [refs]. Over 17 pages, participants can explore the topics in any order with the possibility to write annotations after each segment. With the help of a personal introduction to the study, the booklet prompts to make use of the codes provided. Incorporating graphics from the digital resources tries to invoke reminiscent thoughts while playing with the probe. To manifest the learning process of every participant, the zine can be kept as a present and may leave room for further discussion when addressed during the interview.
\subsubsection{Digital Learning Platform}
When users are prompted to the website referenced in the booklet, further details on the study are revealed. Through minimal design and comprehensive texts, participants can explore the site freely without bumping into frustration that may reflect negatively on their motivation to keep working. Entering the found codes is supposed to be quick and responsive while leaving a sense of mystery. Courses are revealed step by step and visualize a user's progress with their personal entries. \\ \\
Aside from the study, the website includes five main navigation points that function as information sources about the site. A home link leads users directly to the general information on their study including their entry point to their user hub. The editor without a connection to the tasks remains as an opportunity to play with the tool and explore it outside of the scope of the study. An About page references the resources used for creating the website for full transparency. LYRE, the name of the zine, leads to download links for the digital pdf version in case the booklet got lost, or may be the preferred way of reading every chapter. The final link MORE reveals information outside of the study that may help participants to delve deeper into the world of computing if they enjoyed any of the topics referenced in the course. These tips are strictly objective and based on my personal experiences as a developer and artist and are not a part of the main study.
\paragraph{Course Overview}
\paragraph{Personal User Hub}
\paragraph{Task Progression}
\subsubsection{Pseudocode Editor}
The graphical interface is very similar to state-of-the-art online editors like Canva or Plotno. By creating building blocks like texts and images users may place them on a canvas while editing them to their taste. To improve and adjust this functionality to coding practices, coding blocks and animations simulate the unique feeling of writing code. Coding blocks include tabbing, input/output pairs, line numbers, and different types of expression tags. Animations can be attached to each block to bring them to life when running the code. As soon as you push the button, the places blocks will emerge in the order of their layering, play their animations, and visualize the expressions set in the coding blocks. This small step is supposed to give users feedback on their creations, and their submissions may be seen on their respective hub space as personal accomplishments. 
\paragraph{Canvas Interaction}
\paragraph{Pseudoprogram Execution}
\paragraph{Submission and Reflection}
\paragraph{Submission Analysis}
