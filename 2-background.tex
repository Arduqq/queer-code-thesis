\chapter{Background}
% programmierkurse ausgerichtet auf industrie standards
% design der online platformen is von unternehmen gestützt und impliziert informatik als werkzeug für arbeitstools
% kann coden als kunst angewandt werden? feminismus als ansatz zur entkopplung von coding online lehre von industrie
% kreativer umgang mit code thematisiert durch material feminism, händischer auseinandersetzung mit code 
% nutzen von spielerischen handwerken, um komplexe näherzubringen, ohne aufgaben und anforderungen durch industrie vorzugeben
% kreatives arbeiten mit code
% kreatives coden angewandt meistens für kinder - ansatz für erwachsene mit dem thema umzugehen
% erwachsene augen als politisch und sozial - soziale, schwierige fragen zur lehre nutzen, um machtstrukturen aufzudecken
With computing as an increasingly vital part of society, discussions on the technologies involved have been widespread. Its value as a powerful tool to automate tasks, organize social structures, or schedule our daily activities has been helpful not only for companies, but for most people with access to the internet. This infinite space, that would be the home for interactions between these parties with diverging interest in the medium, has created power structures that have to be explored to enforce a virtual life for people online without risking their safety. An access to computing is the ability to create program code, and the education thereof remains to be understood on itself. \\ \\
Coding as a medium opens the door to understand the thinking and the opportunities developers and their employers use to create products that people use in their everyday life. Building a software is an increasingly complex, and costly endeavor with [Quelle für Kosten von Softwareentwicklung 2023]. Consequently, the people observing this trend would want to optimize such processes, educate staff, promote the helpful tools, and look for working force fit for the job. [Quelle zu Software Skills für Full Stack]. Logically, programming training promotes the means of web development, network administration, and teamwork organization. [Google Kurs Beispiel. While relevant and complex, this barely scratches the surface of Computer Science as a research field. [Aktuelle Trends] With these main fields presented, they are measured under its performance on the market which obfuscates the borders between academic and corporate research. If programming itself is so closely tied to its market value, the analysis of its form of education is of high interest. \\ \\
Teaching how to code has evolved over the steady evolution of online structures. With books on programming [Quellen] making room for online courses [Quellen], and corporations providing the skillsets they are seeking for free [Quelle], the resources seem to be plentiful. Online learning platforms like Codecademy, KhanAcademy or Udemy have made a great effort in providing live coding platforms for users to feel more involved in the coding process by solving pre-defined problems, running validity checks, and discussing problems with other learners. With a course curriculum tied to the programming languages learned which feature prominent programming languages on the market [Quelle], the journey of a programmer seems to be set in stone by industry-funded learning platforms [Quelle] and constantly updated frameworks by other developers. What remains to be explored is a discussion of these technologies on a social layer, that is ignored throughout such a journey. Not only are they overshadowed by modern, gamified interfaces, but mastering state-of-the-art frameworks gives no leeway to reflect on your own code. Before dealing with dedicated software modules to safe time, direct confrontation with the basis of programming can reveal where ones interest in soft- and hardware are. \\ \\
As with every medium, code bears the potential for creative expression. Compared to other conventional media like paintings, print, or music, programs are rarely perceived as artistic. [Quelle] Understanding and programming a computer is associated with math, complex theory, and high economic value. Despite the efforts for diversity in STEM, [Quelle] high-paying jobs are still occupied by cis men [Quelle]. It is apparent that not only is the field unappealing in its current industrial, performance-driven nature, but the societal opinion on code and its aesthetics is unattractive. Rarely are creative processes explored in school education and private courses. Emphasizing the capabilities of programming outside of writing functioning websites, robust networks or powerful learning models, is a key factor to reshape the public opinion on code. Through using code syntax as an act of expression, creating software beyond the scope of computing, and including other media to elevate artistic meaning, people may become accustomed to the real nature of programming, demystifying the algorithm. Bridging the gap between knowing of algorithms to understanding them is a key factor in raising awareness to the tangible threats of computing culture. Accordingly, teaching the creative and social aspect of code in computing education becomes crucial. \\ \\ 
To take inspiration from the teachings of traditional media, teaching code may not only rely on processing pre-defined syntax. Pseudocode as an already established representation of code for teaching and publication may become a first step into expressing yourself artistically through your own syntax fit under the scope of computing. By formatting, animating and customizing Pseudocode, multi-medial opportunities arise that may appeal to an audience that is unfamiliar with code.
