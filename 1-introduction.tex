\chapter{Introduction}
% Thema: Programmierung wird gesellschaftlich als kompliziert und mathematisch gelesen auf Grund patriarchaler Strukturen (Parallelen Industrie und Bildungsangebote, Bildungsstrategien von Programmieren, Öffentliche Meinungen, Alternative Formen der Codelehre)
% Fokus: Bildungsstrategien fernab von Live Courses durch Pseudocode, Entsteht ein Lerneffekt auf einer Lernplattform, die ausschließlich Pseudocode und Exploration verwendet?
% Relevanz: Einseitige Beleuchtung komplexer, machtvoller Gebiete der Informatik ohne soziale Stimmen fördert die fehlende Analyse von Machtstrukturen im komplexen Medium Code. Menschen fehlt es an digitaler Literacy für das Hinterfragen des Mediums, welches ihr Leben nicht-transparent bestimmt. Umgehen langer techniklastiger Programmierkurse durch Vermittlung der Grundlagen in praktischen kreativen Aufgaben ohne der Syntaxlehre. Wahrnehmung von Code als auch kreativ und potenziell expressiv.
% Hypothese: Kann ein solcher Programmierkurs zu Lernerfolgen und digitaler Auffassungsgabe führen? 
% Ablauf: Aesthetic Programming Inspiration, Formulierung in einen Kurs ohne Echt-Programmierung, Umsetzung der mulitmedialen Lehre durch Digital-Analoge Probestudy, Zine mit Lehrtexten aus AP, Webseite mit digitalen Resourcen zur Exploration, Programmierumgebung für Journal-ähnliche Programme mit Ziel Pseudocode zu beleben, Kursdurchlauf mit Menschen ohne Programmierhintergrund, Analyse der erstellten Werke, deren Reflexion, und Evaluation des Lernerfolges durch finales Interview.

% Thema
% Online Platformen als Möglichkeit der Lehre - Software as A Service vs Art - Einfluss traditioneller Kunst- und Handwerkslehre in Codelehre - Potenzial liegt in der Erweiterung von Pseudocode
Teaching computer programming in its prominence, is a vital process to familiarize people with creating, maintaining, and improving software. Despite its seemingly complex and mathematical nature, Computer Science remains to be an attractive degree for young people to pursue. [Q] With a high demand of programming labour within the industry [Q] companies release their own educational programs to enable a swift, cheap and engaging experience to welcome young, able and motivated staff. Coding is perceived as lucrative and exclusive to people who understand and memorize the intricacies of modern programming frameworks, while areas outside the commercial scope are rarely explored. Moreover, the increasing influence of the industry on programming education is reflected in code that is written to be efficient and marketable, as well as releasing products that appeal to as many people as possible. This trend not only limits the potential of code, but may be exploited by power dynamics where certain demographics are unfamiliar with the underlying processes. Digital literacy becomes essential for navigating through the infinite virtual landscape, yet its education becomes increasingly difficult when the first resources online will teach the technology of companies. These popular learning tools focus on successive, gamified, online interfaces to teach fundamental coding principles, yet relevant social implications, expressive potential, and power dynamics remain ignored. Acknowledging these factors during learning processes may teach the fun and alternative way of thinking of computer science, but enables critical reflection of new and old technologies. \\ \\
In this thesis, I present a learning platform that focuses on removing executable code from programming education, making full use of the concept of pseudocode. When reading about current discourse on technology, exploring art of various media and reflecting artistically on the newly learned concepts through an alternative representation of a computer program, the platform shall spark interest and bridge the gaps between people and code without relying on industry-driven courses. With the goal to shed light on the creative possibilities and potential threats of this medium, I try to improve digital literacy among people of all age groups. What remains to show is that a programming course on the social and mathematical fundamentals conducted over creative pseudocode will produce effective learning experiences. \\ \\
Based on the work by Soon and Cox [Q], I curated a learning platform that would compile their referenced resources, and lead participants to a graphical editor environment where they would be able to create pseudocode. Through animations, graphics and prose, code can be enhanced to become a form of expression that doesn't require people to be familiar with either such graphical interfaces, or programming syntax. Accompanied by a zine on the critical reflection of computer science basics, participants are free to explore the topics they feel most interested in while reflecting on their own opinions. 