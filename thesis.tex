% \documentclass[german,bachelor,mlu]{webisthesis}
% \documentclass[german,master,buw]{webisthesis}
%
% Non-default programme
% ---------------------
\documentclass[english,master,buw]{webisthesis}\global\thesisprogramme{Computer Science for Digital Media}
% \documentclass[english,master,buw]{webisthesis}\global\thesisfrontpagefaculty{Faculty of Civil Engineering/Faculty of Media}\global\thesisprogramme{Digital Engineering}
% \documentclass[german,bachelor,buw]{webisthesis}\global\thesisprogramme{Informatik\\Schwerpunkt Medieninformatik}
% \documentclass[german,bachelor,buw]{webisthesis}\global\thesisprogramme{Informatik\\Schwerpunkt Security and Data Science}
%
% When you change the language, pdflatex may halt on recompilation.
% Just hit enter to continue and recompile again. This should fix it.


%
% Values
% ------
\ThesisSetTitle{Teaching Queer Code: Educating Computer Science under Social Aesthetics}
\ThesisSetKeywords{masters thesis, queer, computer science} % only for PDF meta attributes
\ThesisSetLocation{Weimar, Germany} 

\ThesisSetAuthor{Artur Solomonik}
\ThesisSetStudentNumber{115715}
\ThesisSetDateOfBirth{10}{01}{1997}
\ThesisSetPlaceOfBirth{Tscherepowez}

% Supervisors should usually be Professors from the candidate's university. A second supervisor is not always needed. 
\ThesisSetSupervisors{Britta Schulte, Prof.\ Dr.\ Eva Hornecker}

\ThesisSetSubmissionDate{31}{2}{2022}

%
% Suggested Packages
% ------------------
\usepackage[sort&compress]{natbib}
%   Allows citing in different ways (e.g., only the authors if you use the
%   citation again within a short time).
%
\usepackage{booktabs}
%    For tables ``looking the right way''.
%
% \usepackage{tabularx}
%    Enables tables with columns that automatically fill the page width.
%
% \usepackage[ruled,algochapter]{algorithm2e}
%    A package for pseudo code algorithms.
%
% \usepackage{amsmath}
%    For tabular-style formatting of mathematical environments.
%

\usepackage{fontawesome}
%    For lots of awesome glyphs: https://mirror.physik.tu-berlin.de/pub/CTAN/fonts/fontawesome/doc/fontawesome.pdf

%
% Commenting (by your supervisor)
% -------------------------------
\usepackage{xcolor}
\usepackage{soul}
\newcommand{\bscom}[2]{%
  % #1 Original text.
  % #2 Replacement text.
    \st{\scriptsize\,#1}{\color{blue}\scriptsize\,#2}%
  }

% Create links in the pdf document
% Hyperref has some incompatibilities with other packages
% Some other packages must be loaded before, some after hyperref
% Additional options to the hyperref package can be provided in the braces [], like in
% \usehyperref[backref] % This will add back references in the bibliography that some people like ... some don't ... so better ask your supervisor ;-)
\usehyperref

\begin{document}
\begin{frontmatter}
\begin{abstract}
Creating software is closely tied to overcoming knowledge gaps between peers, automating busy work and improving code bases. The underlying processes function under a collective effort to maximize profit. With a multitude of ever-growing research tracks, the creative possibilities and social discourses seem diverse and fruitful. This promise, however, is challenging to upkeep as the industrial complex highly relies on the labor of writing code in the most effective ways possible. Technical evolution is rarely discussed on its ethics and politics during education, while the people affected by it do not get the time and resources to build digital fluency. The need for software developers and educational imbalance paint a narrative of the powerful and future-shaping developer. Learning resources closely relate to learning specific programming languages or frameworks, understanding the principles of code delivery, and working effectively in corporate teams. Swiftly, the potential of an unexplored medium such as code becomes an instrument for the industry, rather than subject to critical analysis and playing. What seems to be an opportunity to shape ideas in ways unimagined, becomes a commodity that is ought to be taught by the complex power dynamics of the cisheteropatriarchy. \\ \\
Computer science research groups, art collectives and (cyber-)feminist institutions embrace the idea of queering coding and standing up for the notions of creative processes in technology. Understanding the prevalent power of soft- and hardware and the ones who shape it, individuals can be educated on the focus of computer science by introducing feminist ideas on approaching the medium. Forming a collective consciousness on the impact of the medium code and applying it in technology even outside the generative context enables novel interactions with technology and the people influenced by it. With that in mind, while this thinking saw its implementation in coding projects with children, university courses, teaching resources, public art installations and video games already, digital, independent learning platforms and respective online infrastructure remains mostly unexplored.  \\ \\
In my thesis, I present an online education platform for exploring the principles of computer science, its societal implications and implementations in art, culture and prose. It is to show that by teaching creative code through a feminist gaze, people can critically comprehend the ideas of computer science. Based on the principals of Aesthetic Programming, alongside a corpus of resources from leading research on creative programming and curated guides, I provide an editor environment to build pseudocode through the possibilities of digital internet art. Over the course of two weeks, individuals work on their own programs while reflecting on their works and ideas. Through a probe study, and an interview, the individuals' assessment on their personal growth in the field will reveal the potential on teaching queer code. 
\end{abstract}

\tableofcontents

% \chapter*{Acknowledgements} % optional
% I thank the authors of the webisthesis template for their excellent work!

% \listoffigures % optional, usually not needed

% \listoftables % optional, usually not needed

% \listofalgorithms % optional, usually not needed
%    requires package algorithm2e

% optional: list of symbols/notation (e.g., using the nomencl package) but usually not needed
\end{frontmatter}

\chapter{Introduction}
% Thema: Programmierung wird gesellschaftlich als kompliziert und mathematisch gelesen auf Grund patriarchaler Strukturen (Parallelen Industrie und Bildungsangebote, Bildungsstrategien von Programmieren, Öffentliche Meinungen, Alternative Formen der Codelehre)
% Fokus: Bildungsstrategien fernab von Live Courses durch Pseudocode, Entsteht ein Lerneffekt auf einer Lernplattform, die ausschließlich Pseudocode und Exploration verwendet?
% Relevanz: Einseitige Beleuchtung komplexer, machtvoller Gebiete der Informatik ohne soziale Stimmen fördert die fehlende Analyse von Machtstrukturen im komplexen Medium Code. Menschen fehlt es an digitaler Literacy für das Hinterfragen des Mediums, welches ihr Leben nicht-transparent bestimmt. Umgehen langer techniklastiger Programmierkurse durch Vermittlung der Grundlagen in praktischen kreativen Aufgaben ohne der Syntaxlehre. Wahrnehmung von Code als auch kreativ und potenziell expressiv.
% Hypothese: Kann ein solcher Programmierkurs zu Lernerfolgen und digitaler Auffassungsgabe führen? 
% Ablauf: Aesthetic Programming Inspiration, Formulierung in einen Kurs ohne Echt-Programmierung, Umsetzung der mulitmedialen Lehre durch Digital-Analoge Probestudy, Zine mit Lehrtexten aus AP, Webseite mit digitalen Resourcen zur Exploration, Programmierumgebung für Journal-ähnliche Programme mit Ziel Pseudocode zu beleben, Kursdurchlauf mit Menschen ohne Programmierhintergrund, Analyse der erstellten Werke, deren Reflexion, und Evaluation des Lernerfolges durch finales Interview.

% Thema
% Online Platformen als Möglichkeit der Lehre - Software as A Service vs Art - Einfluss traditioneller Kunst- und Handwerkslehre in Codelehre - Potenzial liegt in der Erweiterung von Pseudocode
Teaching computer programming in its prominence, is a vital process to familiarize people with creating, maintaining, and improving software. Despite its seemingly complex and mathematical nature, Computer Science remains to be an attractive degree for young people to pursue. [Q] With a high demand of programming labour within the industry [Q] companies release their own educational programs to enable a swift, cheap and engaging experience to welcome young, able and motivated staff. Coding is perceived as lucrative and exclusive to people who understand and memorize the intricacies of modern programming frameworks, while areas outside the commercial scope are rarely explored. Moreover, the increasing influence of the industry on programming education is reflected in code that is written to be efficient and marketable, as well as releasing products that appeal to as many people as possible. This trend not only limits the potential of code, but may be exploited by power dynamics where certain demographics are unfamiliar with the underlying processes. Digital literacy becomes essential for navigating through the infinite virtual landscape, yet its education becomes increasingly difficult when the first resources online will teach the technology of companies. These popular learning tools focus on successive, gamified, online interfaces to teach fundamental coding principles, yet relevant social implications, expressive potential, and power dynamics remain ignored. Acknowledging these factors during learning processes may teach the fun and alternative way of thinking of computer science, but enables critical reflection of new and old technologies. \\ \\
In this thesis, I present a learning platform that focuses on removing executable code from programming education, making full use of the concept of pseudocode. When reading about current discourse on technology, exploring art of various media and reflecting artistically on the newly learned concepts through an alternative representation of a computer program, the platform shall spark interest and bridge the gaps between people and code without relying on industry-driven courses. With the goal to shed light on the creative possibilities and potential threats of this medium, I try to improve digital literacy among people of all age groups. What remains to show is that a programming course on the social and mathematical fundamentals conducted over creative pseudocode will produce effective learning experiences. \\ \\
Based on the work by Soon and Cox [Q], I curated a learning platform that would compile their referenced resources, and lead participants to a graphical editor environment where they would be able to create pseudocode. Through animations, graphics and prose, code can be enhanced to become a form of expression that doesn't require people to be familiar with either such graphical interfaces, or programming syntax. Accompanied by a zine on the critical reflection of computer science basics, participants are free to explore the topics they feel most interested in while reflecting on their own opinions. 
\chapter{Background}
% programmierkurse ausgerichtet auf industrie standards
% design der online platformen is von unternehmen gestützt und impliziert informatik als werkzeug für arbeitstools
% kann coden als kunst angewandt werden? feminismus als ansatz zur entkopplung von coding online lehre von industrie
% kreativer umgang mit code thematisiert durch material feminism, händischer auseinandersetzung mit code 
% nutzen von spielerischen handwerken, um komplexe näherzubringen, ohne aufgaben und anforderungen durch industrie vorzugeben
% kreatives arbeiten mit code
% kreatives coden angewandt meistens für kinder - ansatz für erwachsene mit dem thema umzugehen
% erwachsene augen als politisch und sozial - soziale, schwierige fragen zur lehre nutzen, um machtstrukturen aufzudecken
\section{}
With computing as an increasingly vital part of society, discussions on the technologies involved have been widespread. Its value as a powerful tool to automate tasks, organize social structures, or schedule our daily activities has been helpful not only for companies, but for most people with access to the internet. This infinite space, that would be the home for interactions between these parties with diverging interest in the medium, has created power structures that have to be explored to enforce a virtual life for people online without risking their safety. An access to computing is the ability to create program code, and the education thereof remains to be understood on itself. \\ \\
Coding as a medium opens the door to understand the thinking and the opportunities developers and their employers use to create products that people use in their everyday life. Building a software is an increasingly complex, and costly endeavor with [Quelle für Kosten von Softwareentwicklung 2023]. Consequently, the people observing this trend would want to optimize such processes, educate staff, promote the helpful tools, and look for working force fit for the job. [Quelle zu Software Skills für Full Stack]. Logically, programming training promotes the means of web development, network administration, and teamwork organization. [Google Kurs Beispiel. While relevant and complex, this barely scratches the surface of Computer Science as a research field. [Aktuelle Trends] With these main fields presented, they are measured under its performance on the market which obfuscates the borders between academic and corporate research. If programming itself is so closely tied to its market value, the analysis of its form of education is of high interest. \\ \\
Teaching how to code has evolved over the steady evolution of online structures. With books on programming [Quellen] making room for online courses [Quellen], and corporations providing the skillsets they are seeking for free [Quelle], the resources seem to be plentiful. Online learning platforms like Codecademy, KhanAcademy or Udemy have made a great effort in providing live coding platforms for users to feel more involved in the coding process by solving pre-defined problems, running validity checks, and discussing problems with other learners. With a course curriculum tied to the programming languages learned which feature prominent programming languages on the market [Quelle], the journey of a programmer seems to be set in stone by industry-funded learning platforms [Quelle] and constantly updated frameworks by other developers. What remains to be explored is a discussion of these technologies on a social layer, that is ignored throughout such a journey. Not only are they overshadowed by modern, gamified interfaces, but mastering state-of-the-art frameworks gives no leeway to reflect on your own code. Before dealing with dedicated software modules to safe time, direct confrontation with the basis of programming can reveal where ones interest in soft- and hardware are. \\ \\
As with every medium, code bears the potential for creative expression. Compared to other conventional media like paintings, print, or music, programs are rarely perceived as artistic. [Quelle] Understanding and programming a computer is associated with math, complex theory, and high economic value. Despite the efforts for diversity in STEM, [Quelle] high-paying jobs are still occupied by cis men [Quelle]. It is apparent that not only is the field unappealing in its current industrial, performance-driven nature, but the societal opinion on code and its aesthetics is unattractive. Rarely are creative processes explored in school education and private courses. Emphasizing the capabilities of programming outside of writing functioning websites, robust networks or powerful learning models, is a key factor to reshape the public opinion on code. Through using code syntax as an act of expression, creating software beyond the scope of computing, and including other media to elevate artistic meaning, people may become accustomed to the real nature of programming, demystifying the algorithm. Bridging the gap between knowing of algorithms to understanding them is a key factor in raising awareness to the tangible threats of computing culture. Accordingly, teaching the creative and social aspect of code in computing education becomes crucial. \\ \\ 
To take inspiration from the teachings of traditional media, teaching code may not only rely on processing pre-defined syntax. Pseudocode as an already established representation of code for teaching and publication may become a first step into expressing yourself artistically through your own syntax fit under the scope of computing. By formatting, animating and customizing Pseudocode, multi-medial opportunities arise that may appeal to an audience that is unfamiliar with code.

\chapter{Methodology}
In this chapter, I will reference the work by \cite{soon:2020} to refer to the source of information that participants would study. This study draws from their writings on computing fundamentals and sociopolitical interpretations thereof, and their acknowledged references that were extended through further explanations and other entries.
\section{Design}
% Why is this a probe study?
The main study is conducted as an inspirational probe study.  As described by \citep{mattelmäki:2005}, my probes are of inspirational type and are meant to provoke experiences that may enable open reflection on the sensitive, social topics of computing in society. For this research, my probes are defined as such by its fundamental properties. A kit consisting of a zine, a website with a personal participants' place, and an editor give room for exploratory opportunities the encourage individual and ambivalent confrontation with the topic. For the probes are meant to showcase the creative possibilities for both coders and educators, the study sets an exploratory goal of rethinking the role of computing outside the scope of the industry. By moving the study into learning environments chosen autonomously by the participants, the probes become a part of the users' subjective world as the zine may be taken anywhere, and programming tasks are encouraged to be personal and undefined. The resulting program collages are artifacts of the self-documentation process of reading, note-taking, designing, and reflecting. The computer as a medium for creative self-documentation still remains overshadowed in probe studies by pen and paper, tablets, or mobile input technology. The internet itself as a medium for self-expression provides opportunities for users to log their experiences with probes.  \\ \\
% Why probes?
Choosing this form of a free, yet tangible concept is motivated by the active involvement of the users. Giving participants the building blocks to explore supplementary material like a zine alongside a digital canvas, may enforce more intimate and playful experiences compared to a sequential completion of artificial problems. The open nature of probes encourages strategizing a personal workflow and treats a programming course not as a lecture of topics, but an exploration. As an opportunity to move away from the linear style that sees its implementation in modern day programming courses, probe studies leave room for creative involvement that may even give participants a sense of pride in the creative work they produced. Elevating the probe from a solely analogue form to a digital workspace, may provide further insight on the creative potential of digital media in a playful context and the definition of probe studies. Where digital programming courses lack in tangibility of the code work, the probe can be seen as a gift and real representation of their efforts. \\
% How to overcome the problems of probes? 
As noted by \citeauthor{mattelmäki:2005}, probe studies come with their own difficulties as their ambiguity may frustrate and confuse participants. Providing users with numerous ways of interacting with the probes, like note-taking in the zine, clicking through resources, using different building blocks for the collages, and finding ways to overcome its limitations, may give questioning users of catching momentum when trying out the different interactions. Furthermore, removing pressure in the form of strict deadlines may alleviate stress. Providing inspirational aid at all times to varying extent while having initial rapport with participants may also help users to overcome the hurdle to create freely. \\ \\

\section{Participants}
\section{Materials}
% What does the probe look like?
The probe is both analogue and digital in form of a written zine containing the learning articles, and a hub space on a website to track your progress. To give the platform recognizability and a homely feeling, the design of the probe is cohesive throughout and is given character through custom graphics, icons and style. The title Assemblyre is a direct reference to the programming language Assembler, a complex programming language that directly communicates with the computer, and a lyre, which puts emphasis on the effort of highlighting the inherent creative potential of computing and its acceptance. The motivation to change between analogue and digital comes from the reference of codes throughout the zine. User codes remain unique and serve as a way to identify the work and their submitted task, while the codes of the courses are the same since there are no variations of the respective chapters. \\ \\
Every chapter consists of the excerpt from the guide on Aesthetic Programming by \cite{soon:2020}, the respective course code, open questions about the topic discussed, and a list of links to digital resource with short summaries. The excerpts were chosen according to the scope of this study. To avoid confusion, programming tasks with P5js, reflections on their course, and paragraphs relying on the involvement in the original course have been removed to serve the purpose of being publishable in a zine. With that in mind, I cited their work without obfuscating meaning and provided context where it may have been unclear. The goal of the zine would not be the active confrontation with the complex details, but the acknowledgment of their existance, to spark interest and leave the investigation up to the participants.
\section{Procedure}
\section{Analysis}
%My probe study on the field of queer code takes inspiration from supporting technical education with a highly critical look on its social aspect. By advocating for ethics and politics in the technical and industry-driven curriculum of Computer Science, I want to analyze the potential of aesthetic programming. Aesthetic programming combines the creative and unique nature of the medium code and clearly states complex power dynamics during the process. As these matters must be communicated and supported in sophisticated programming courses, I provide an approach that circumvents the need for machines that are able to code, or the amount of time necessary to realize bigger projects. By using pseudocode to express yourself creatively while employing the concepts of Computer Science, participants partake in a study of two weeks. Here, they read about current arguments surrounding basic concepts as data, abstraction or machine learning while creating their own art pieces. These a comparable to journaling techniques that are enhanced through the possibilities of animating objects, or styling them accordingly while creating a sense of running your code. The results will give further insight on the effectiveness of communicating these very complex, yet vital concepts, while encouraging people to play with the ever-evolving medium of program code.

%What kind of study is it?
New content on the website is unlocked by the users as they progress the reading of the zine. By typing a new code on the platform, digital resources and open questions of interpretation are unlocked that are supposed to spark inspiration to the participants. Largely inspired by the resources provided in the Aesthetic Programming course, the references are realized in simple button clicks alongside a short preface. Instead of researching links from a long list of literature, the more visual and easily digestible content provides participants with an easier experience. Accordingly, they may invest their time to their liking and explore the things they find the most interesting instead of reading through long articles without personal connection.
The additional resources are used as inspiration for their programming task and will then navigate to the respective editor to start implementing. User may submit their work over their hub, and will see the progress in the form of their finished work. Every submission comes with a short description of the participant’s vision, their artistic choices, and a reflection on the hypothetical power dynamics of the program. The submissions are then showcased to the participants over their hub.

To provide context to the art the participants produce, a short assessment of their experience and opinion with programming is conducted in form of a survey. Here, the overall demographic, their societal perception of programming, and current knowledge on the topic is asked to aid the analysis of the probe study.

A supporting interview will be conducted in a semi-structured form where participants will be asked about their experience with the learning platform. If they are comfortable, participants may talk about the code pieces they created to assess their thought processes, and whether their ideas were realized on the learning platform accordingly. Talking to participants directly will also provide valuable information on whether the course was able to communicate the topics appropriately. 

How is the probe study conducted?
Participants are presented with eight topics of computer programming, and will create at least five art pieces through the graphical interface on the website. Over the course of a minimum of two weeks, they can read the information in the zine, unlock the digital resources, and create a code piece of whatever topic they wish. For additional inspiration, questions are provided as thought impulses.

How is the probe study analyzed?
The results come in form of graphical pieces resembling customized, expressive pseudocode, as well as the individual reflections on the different topics. The consecutive executions of the tasks may show an interest in the topics, as well as a potential learning effect of the concept of pseudocode as well as working with the graphical interface. All pieces provide insight on possible interpretations of the principles presented in the course and open questions for further discussion. Additionally, commentary on the graphical interface is valuable to exploring the field of pseudocode as an art form.

Why is a probe study valuable for assessing a learning effect of a learning platform?
A probe study is valuable to the analysis of a learning platform, as it encourages the participants to partake in the numerous resources and explorable environments provided. Reading articles, unlocking custom content and creating playful designs is supposed to aid their willingness to learn in a non-exploitative context. In the sense of material feminism, getting involved with the medium physically is crucial to learning experiences.   Transferring participants between analogue and digital media supports the idea that programming is not an exclusively digital craft. Print media gives room for experimentation, note-taking and drawing, whereas digital interfaces provide aesthetic feedback that encourages exploration.
The study is conducted with seven participants that are all part of my friend groups. A small number of samples was chosen to have a better picture of the individual work and qualitative data. I chose people who are of varying backgrounds in occupation, gender, and academic levels. This opens more possibilities of artistic expression as all these factors influence different life experiences. Since the study is supposed to analyze the learning of programming principles, participants have little to no knowledge of coding or computer science. All the participants are in the age group between 21 and 27 which is coincidental and not a part of the selection process. 
The users are provided with a booklet designed and curated by me. The included texts and graphics all originate from an extensive course on Aesthetic Programming.  Digital resources were chosen by me with the motivation to add to the multitude of links curated by Soon and Cox. Links and art installations were added with the help of the Cyberfeminism Index  and the browser-based game section of Itch . Recorded data comes in the form of a JSON file that is used to build the graphical media through the study’s website.  The submission takes place on the SoSci servers and will then be stored both on the site, and a hard drive up until the end of the thesis defense.

The graphical interface is very similar to state-of-the-art online editors like Canva or Plotno. By creating building blocks like texts and images users may place them on a canvas while editing them to their taste. To improve and adjust this functionality to coding practices, coding blocks and animations simulate the unique feeling of writing code. Coding blocks include tabbing, input/output pairs, line numbers, and different types of expression tags. Animations can be attached to each block to bring them to life when running the code. As soon as you push the button, the places blocks will emerge in the order of their layering, play their animations, and visualize the expressions set in the coding blocks. This small step is supposed to give users feedback on their creations, and their submissions may be seen on their respective hub space as personal accomplishments. 
The study is communicated both in person and through online communication over e-mail. In a personal meeting, each participant is handed their probe kit and is instructed in the details of the study. In a collective message, the details are again described and the website is shared. During the study, participants can message me personally in case any technical difficulties come up. In case creative problems arise, I provide further aid in form of pseudocode examples and further explanations on the topics. In case of time issues, the individual study deadline of two weeks are postponed. 

Interviews are conducted in a personal meeting in a familiar setting like my or their home. The interview is recorded on audio and is then transcribed to analyze the participants’ opinion on the information provided, and their learning efforts.
For the analysis, the graphical interface’s import functionality is used to view the code pieces created by the participants. The tool enables the deduction of what features were especially useful for conveying certain types of programming concepts. The provided self-reflection by users facilitates the interpretation of the results. Findings are then finalized through individual questions during the interview.

\section{Mixed-Media Probe Study}
\subsection{Programming Course Design}
\subsubsection{Resource Collection}
\subsubsection{Course Structure}
\subsection{Analogue Zine}
\subsection{Digital Learning Platform}
\subsubsection{Course Overview}
\subsubsection{Personal User Hub}
\subsubsection{Task Progression}
\subsection{Pseudocode Editor}
\subsubsection{Canvas Interaction}
\subsubsection{Pseudoprogram Execution}
\subsubsection{Submission and Reflection}
\subsubsection{Submission Analysis}
\chapter{Results}
\chapter{Discussion}
\chapter{Threats to Validity and Future Work}

\appendix
\chapter{My First Appendix}
This was just missing.



% Bibliography
\bibliographystyle{plainnat} % requires package natbib. An alternative is apalike
\bibliography{literature}    % load file literature.bib

\end{document}

